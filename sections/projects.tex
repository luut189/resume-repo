% !TEX root = resume.tex

\section{Projects}
\resumeSubHeadingListStart
\resumeProjectHeading
{\href{https://github.com/luut189/anime-discuss}{\textbf{AniDis - Anime Discussion Platform}} $|$
  React, TypeScript, Node.js, Express, MongoDB, Docker}{}
\resumeItemListStart
\resumeItem{Developed \textbf{AniDis}, a MERN-based anime discussion platform with real-time updates for \textbf{1,000+} titles and threaded comments supporting unlimited nested replies.}
\resumeItem{Automated data ingestion and trend updates with scheduled backend jobs, ensuring daily content freshness and scalable performance.}
\resumeItem{Containerized deployment with Docker/Docker Compose, environment-based configuration, and CI/CD-ready Git workflows; self-hosted with Cloudflare Tunnel.}
\resumeItem{Implemented security best practices and optimized frontend rendering for high-performance discussions.}
\resumeItemListEnd

\resumeProjectHeading
{\href{https://github.com/luut189/kyzen}{\textbf{Kyzen - 2D Game Engine}} $|$
  Java, LWJGL, Maven, OpenGL}{}
\resumeItemListStart
\resumeItem{Developed a 2D game engine with batch rendering, optimizing the rendering of multiple objects in a single batch for improved performance by up to \textbf{60\%} when handling \textbf{1000+} objects per frame}
\resumeItem{Utilized OOP principles to create scalable software, implementing design patterns like Builder and Singleton to ensure modularity and maintainability}
\resumeItem{Implemented an Entity-Component System (ECS), providing a modular and extendable architecture for flexible game object composition}
\resumeItem{Integrated texture atlas support, enabling efficient texture management for sprites and tiles, reducing draw calls by over \textbf{70\%}}
\resumeItemListEnd
\resumeSubHeadingListEnd